\documentclass{sig-alternate-05-2015}
\begin{document}
\title{The Report of ADM Project 1} 
\numberofauthors{1} 
\author{ 
\alignauthor
       TEMPLATE\\
       \affaddr{TEMPLATE}\\
       \email{TEMPLATE@TEMPLATE}
 } 
\date{24 March 2018}
 
\maketitle

\begin{abstract}
In this report, we establish a cluster with MPI environment.
Hello world and calculating pi are performed.

\end{abstract}
%\keywords{ADM; MPI}
\section{Introduction} 
In this report, we establish a cluster with MPI environment.
 To run the program on different hosts, the following steps should be done:
 \begin{enumerate}
 \item Each host generates a pair of public key and private key. To provide SSH connections without passwords, each host adds others' public keys to authorized keys.
 \item The host, which is selected as the server, should create a folder which can be shared to other hosts. Other nodes should mount to the folder.
 \end{enumerate}
The remainder of this article is structured as follows: 
In Section \ref{hello}, the new functions we used in the program of hello world are introduced. 
In Section \ref{pi}, the principle of calculating $pi$  and the new functions we used in the program of calculating pi are introduced.
In Section \ref{ana}, we analyze the performance of our programs with different number of hosts.
\section{MPI Hello World}\label{hello}
\begin{enumerate}
 \item MPI\_Init
 \item MPI\_Comm\_rank
 \item MPI\_Get\_processor\_name
 \item MPI\_Barrier
 \item MPI\_Wtime
 \item MPI\_Finalize
\end{enumerate}
 

\section{Calculate $\pi$}\label{pi}
We have the equation
\begin{equation}\label{arcpi}
 \int_{x=0}^{1}\frac{4}{1+x^2}dx=\pi.
\end{equation}
Inspired from \eqref{arcpi}, the principle of calculating $\pi$ approximately in our program is to calculate
\begin{equation}
 \sum\limits_{i=1}^{n}\frac{4}{1+\left(\frac{i-0.5}{n}\right)^2},
\end{equation}
where $n$ is used to control the accuracy.

\begin{enumerate}
 \item MPI\_Bcast
 \item MPI\_Reduce
\end{enumerate}

How do we divide the tasks?
 
\section{Performance Analysis}\label{ana}
\subsection{Hello World}
Simulate the result with different host number.

\subsection{Calculate $\pi$}
Simulate the result with Different host number and Different scale of $n$.
%\section{Conclusion} 
%\bibliographystyle{abbrv}
%\bibliography{reference}  
%\appendix 
\end{document}
